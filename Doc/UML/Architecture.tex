\documentclass[a4paper,10pt]{article}
\usepackage[utf8]{inputenc}
\usepackage[T1]{fontenc}
\usepackage{tikz}
\usepackage{tikz-uml} 
\usepackage{amsmath}


%opening
\title{}
\author{}

\begin{document}

%\maketitle

%\begin{abstract}
%\end{abstract}

%\section{}

\begin{tikzpicture}
\begin{umlpackage}{Sample description} 
% Code from official documentation goes here...
\umlinterface{ISample}{
}{
  \umlvirt{+ clone() : ISample*} \\
  \umlvirt{+ createDWBASimulation() : DWBASimulation*}
}
\umlclass[y=-4]{MultiLayer}{
  -- m\_layers : std::vector<Layer *> \\
  -- m\_interfaces : std::vector<LayerInterface *>
}{
  + getNumberOfLayers() : size\_t \\
  + getNumberOfInterfaces() : size\_t \\
  + addLayer(const Layer \&layer) : void
}
\umlclass[x=8,y=-4]{Layer}{
  -- mp\_material : IMaterial* \\
  -- m\_thickness : double
}{
  + getThickness() : double \\
  + setThickness(double thickness) : void
}
\umlinherit[geometry=-|]{MultiLayer}{ISample}
\umlinherit[geometry=|-]{Layer}{ISample}
\umluniassoc[geometry=--, mult2=n]{MultiLayer}{Layer}
\end{umlpackage}
\end{tikzpicture}

\begin{tikzpicture}
\begin{umlpackage}{Simulation Data}
\umlclass{Experiment}{
  -- mp\_sample : ISample* \\
  -- mp\_sample\_builder : ISampleBuilder* \\
  -- m\_detector : Detector \\
  -- m\_beam : Beam \\
  -- m\_intensity\_map : OutputData<double> \\
  -- m\_sim\_params : SimulationParameters
}{
  \umlvirt{+ clone() : Experiment*} \\
  \umlvirt{+ runSimulation() : void} \\
  \umlvirt{+ normalize() : void}
}
\umlemptyclass[x=7, y=0]{ISample}
\umlemptyclass[x=7, y=-2]{Detector}
\umlemptyclass[x=7, y=-4]{Beam}
\umlemptyclass[x=7, y=-6]{SimulationParameters}
\umlemptyclass[y=-4]{GISASExperiment}
\umluniassoc[geometry=|-, anchor1=0]{Experiment}{ISample}
\umluniassoc[geometry=|-, anchor1=0]{Experiment}{Detector}
\umluniassoc[geometry=|-, anchor1=0]{Experiment}{Beam}
\umluniassoc[geometry=|-, anchor1=0]{Experiment}{SimulationParameters}
\umlinherit[geometry=--]{GISASExperiment}{Experiment}

\umlnote[y=-6.5, width=6cm]{GISASExperiment}{
  The ``runSimulation()'' method retrieves an ISimulation object
  from the topmost ISample object and calls its ``run()'' method 
  to perform the actual computations.
}
  
\end{umlpackage}


\end{tikzpicture}

\end{document}
