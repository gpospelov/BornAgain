\newpage
\chapter{Installation} \SecLabel{Installation}

\BornAgain\ is intended to work on x86/x86\_64 Linux, Mac OS X and Windows operating systems. It was successfully compiled and tested on
\begin{itemize}
\item Microsoft Windows 7 64-bit, Windows 8 64-bit
\item Mac OS X 10.8 (Mountain Lion)
\item OpenSuse 12.3 64-bit
\item Ubuntu 12.10, 13.04 64-bit
\item Debian 7.1.0, 32-bit, 64-bit
\end{itemize}

At the moment we support build and installation from source on Unix Platforms 
(Linux, Mac OS) and
installation using binary installer package on MS Windows 7,8 (see
\SecRef{InstallationUnix} and \SecRef{InstallationWindows}).
In the next releases we are planning to provide binary installers for 
Mac OS X and Debian.

We welcome user feedback and/or bug reports related to they installation experience
via \url{http://apps.jcns.fz-juelich.de/redmine/projects/bornagain/issues}


\section{Building and installing on Unix Platforms.} \SecLabel{InstallationUnix}


\BornAgain\ uses \Code{CMake} to configure a build system for compiling and installing the framework. There are three major steps to building \BornAgain\ :
\begin{enumerate}[1.]
\item Acquire required third-party libraries.
\item Get \BornAgain\ source code.
\item Use \Code{CMake} to build and install software.
\end{enumerate}
The remainder of this section explains each step in detail.

\subsection{Third-party software.}
To successfully build \BornAgain\ a number of prerequisite packages must be installed.

\begin{itemize}
\item compilers: clang  versions $\geq 3.1$ or GCC versions $\geq 4.2$
\item cmake ($\geq 2.8.3$)
\item boost library ($\geq 1.48$)
\item GNU scientific library ($\geq 1.15$)
\item fftw3 library ($\geq 3.3$)
\item python ($\geq 2.7$, $< 3.0$), python-devel, python-numpy-devel
\end{itemize}
\vspace*{2mm}

Other packages are optional
\begin{itemize}
\item ROOT framework (adds several additional fitting algorithms to \BornAgain)
\item python-matplotlib (allows to run usage examples with graphics)
%\item Eigen3 library ($\geq 3.1.0$)
\end{itemize}

All required packages can be easily installed on most Linux distributions using the system's package
manager. Below we give a few examples for several selected operation systems. Please note,
that other distributions (Fedora, Mint, etc) may have different commands for invoking the package manager and slightly different names of packages (like ``boost'' instead of ``libboost'' etc). Besides that, the installation should be very similar.
\vspace*{3mm}


% ---------------
%  Ubuntu 13.04
% ---------------
\noindent
{\large\bf Ubuntu (12.10, 13.04), Debian (7.1)} \newline
Installing required packages
\begin{lstlisting}[language=shell, style=commandline]
sudo apt-get install git cmake libgsl0-dev libboost-all-dev libfftw3-dev python-dev python-numpy
\end{lstlisting}

\noindent
Installing optional packages
\begin{lstlisting}[language=shell, style=commandline]
sudo apt-get install libroot-* root-plugin-* root-system-* ttf-root-installer libeigen3-dev python-matplotlib python-matplotlib-tk
\end{lstlisting}
\vspace*{3mm}


% ---------------
%  OpenSuse 12.3
% ---------------
\noindent
{\large\bf OpenSuse 12.3} \newline
Adding ``scientific'' repository 
\begin{lstlisting}[language=shell, style=commandline]
sudo zypper ar http://download.opensuse.org/repositories/science/openSUSE_12.3 science
\end{lstlisting}

\noindent
Installing required packages
\begin{lstlisting}[language=shell, style=commandline]
sudo zypper install git-core cmake gsl-devel boost-devel fftw3-devel python-devel python-numpy-devel
\end{lstlisting}

\noindent
Installing optional packages
\begin{lstlisting}[language=shell, style=commandline]
sudo zypper install libroot-* root-plugin-* root-system-* root-ttf libeigen3-devel python-matplotlib
\end{lstlisting}
\vspace*{3mm}



% ---------------
%  MacOS 10.8
% ---------------
\noindent
\noindent
{\large\bf Mac OS X 10.8} \newline
To simplify the installation of third party open-source software on a Mac OS X system we recommend the use of \Code{MacPorts} package manager. 
The easiest way to install MacPorts is by downloading the \Code{dmg} 
from \url{www.macports.org/install.php} and running the system's installer.
After the installation new command ``\Code{port}'' will be available in terminal window of your Mac. \

\noindent
Installing required packages
\begin{lstlisting}[language=shell, style=commandline]
sudo port -v selfupdate
sudo port install git-core cmake
sudo port install fftw-3 gsl
sudo port install boost -no_single-no_static+python27 


\end{lstlisting}

\noindent
Installing optional packages
\begin{lstlisting}[language=shell, style=commandline]
sudo port install py27-matplotlib py27-numpy py27-scipy
sudo port install root +fftw3+python27
sudo port install eigen3
\end{lstlisting}




\subsection{Getting source code}
\BornAgain\ source can be downloaded at \url{http://apps.jcns.fz-juelich.de/BornAgain}
and unpacked with
\begin{lstlisting}[language=shell, style=commandline]
tar xfz bornagain-<version>.tar.gz
\end{lstlisting}

\noindent
Alternatively one can obtain \BornAgain\ source from our public Git repository.
\begin{lstlisting}[language=bash, style=commandline]
git clone git://apps.jcns.fz-juelich.de/BornAgain.git 
\end{lstlisting}
\vspace*{3mm}


\noindent
{\bf\large More about Git} \\
Our Git repository holds two main branches called ``master'' and ``develop''. We consider ``master''
branch to be the main branch where the source code of HEAD always reflects latest stable release. \Code{git clone} command shown above
\begin{enumerate}[1.]
\item gives you a source code snapshot corresponding to the latest stable release,
\item automatically sets up your local master branch to track our remote master branch, 
so you will be able to fetch changes from the remote branch at any time using ``git pull'' command.
\end{enumerate}

Master branch is updating approximately once per month.
% that reflects our release cycle.
The second branch, ``develop'' branch, is a snapshot of the current development.
This is where any automatic nightly builds are built from. The develop branch is
always expected to work, so to get the most recent features one can switch source code to it by
\begin{lstlisting}[language=bash, style=commandline]
cd BornAgain
git checkout develop
git pull
\end{lstlisting}
\vspace*{3mm}



\subsection{Building and installing the code}

\BornAgain\ should be build using \Code{CMake} cross platform build system. 
Having third-party libraries installed on the system and \BornAgain\ source code acquired as was explained in
previous sections, type build commands
\begin{lstlisting}[language=bash, style=commandline]
mkdir <build_dir>
cd <build_dir>
cmake  -DCMAKE_INSTALL_PREFIX=<install_dir> <source_dir>
make
\end{lstlisting}
\vspace*{3mm}

Here \Code{<source\_dir>} is the name of directory, where \BornAgain\ source code has been
copied, \Code{<install\_dir>} is the directory, where user wants  the package
to be installed, and \Code{<build\_dir>} is the directory where building will occur.

\MakeRemark{About \Code{CMake}}{
\\Having dedicated directory \Code{<build\_dir>} for build process
is recommended by \Code{CMake}. That allows several builds with different compilers/options from the same source and keeps source directory clean from build remnants. \\
}


Compilation process invoked by the command ``make'' lasts about 10 min for an average laptop
of 2012 edition. On multi-core machines the compilation time  can be decreased by invoking command
``make'' with the parameter ``make -j[N]'', where N is the number of cores.

Running functional tests is an optional but recommended step. Command ``make check''
will compile several additional tests and run them one by one. Every test contains
the simulation of a typical GISAS geometry and the comparison on numerical level of simulation results with reference files. Having 100\% tests passed ensures that your local installation
is correct.
\begin{lstlisting}[language=bash, style=commandline]
make check
...
100% tests passed, 0 tests failed out of 26
Total Test time (real) = 89.19 sec
[100%] Build target check
\end{lstlisting}
\vspace*{3mm}


The last command ``make install'' copies compiled libraries and some usage examples
into  the installation directory.
\begin{lstlisting}[language=bash, style=commandline]
make install
\end{lstlisting}


\subsubsection{Troubleshooting}

In the case of complex system setup, with variety of libraries of different versions 
scattered across multiple places (\Code{/opt/local}, \Code{/usr} etc.),
you may want to help \Code{CMake} to find libraries in proper place. 
In example below
two system variables are defined to force \Code{CMake} to prefer libraries
found in \Code{/opt/local} to other places.
\begin{lstlisting}[language=bash, style=commandline]
export CMAKE_LIBRARY_PATH=/opt/local/lib:$CMAKE_LIBRARY_PATH
export CMAKE_INCLUDE_PATH=/opt/local/include:$CMAKE_INCLUDE_PATH
\end{lstlisting}


\subsection{Running first simulation}

In your installation directory you will find
\begin{lstlisting}[language=bash, style=commandline]
./include - header files for compilation of your C++ program
./lib - libraries to import into python or link with your C++ program
./Examples - directory with examples
\end{lstlisting}

Run your first example and enjoy first BornAgain simulation plot.
\begin{lstlisting}[language=bash, style=commandline]
cd <install_dir>/Examples/python/ex001_CylindersAndPrisms
python CylindersAndPrisms.py
\end{lstlisting}




\section{Installing on Windows Platforms.} \SecLabel{InstallationWindows}


\noindent
{\bf Step I: $~$ installing third party software} \newline
The current version of \BornAgain\ requires \Code{Python, numpy, matplotlib} 
to be installed on the system. If you don't have them already installed,
you can use \Code{PythonXY} installer
at \url{https://code.google.com/p/pythonxy} which, with default installation options, will contain at least these three packages.
The user has to download and install this package before proceeding with
\BornAgain\ installation.
\vspace*{2mm}

\noindent
{\bf Step II: $~$ using installation package } \newline
The Windows installation package can be downloaded from \url{http://apps.jcns.fz-juelich.de/BornAgain}.
Double click it to start the installation process, then follow the instructions.
\vspace*{2mm}

\noindent
{\bf Step IV: $~$ running example} \newline
Run an example by double-clicking on the python script located in the \BornAgain\ installation directory:
\begin{lstlisting}[language=shell, style=commandline]
python C:/BornAgain-0.9.1/Examples/python/ex001_CylindersAndPrisms/CylindersAndPrisms.py
\end{lstlisting}


%\MakeRemark{Compiling on Windows}{
%Compilation of \BornAgain\ from source on Windows using Microsoft Visual Studio is %possible, although not easy. Build instructions can be provided on request.
%}



