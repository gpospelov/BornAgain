\chapter{Formfactors}

\begin{itemize}
\item Parallelepiped
\item Pyramid, \SecRef{Pyramid} 
\item Cylinder
\item Cone
\item Prism3
\item Tethraedron
\item Prism6
\item Cone6
\item Cut-off sphere, \SecRef{Sphere} 
\item Cybooctaedron
\item Facetted sphere  
\item Full sphere, \SecRef{FullSphere} 
\item Full spheroid 
\item Box
\item Anisotropic pyramid
\item Ellipsoid
\item Anisotropic hemi-spheroid
\item Spheroid
\end{itemize}

\newpage
\section{Formfactor Cut-off Sphere}\SecLabel{Sphere}
\subsection{Real-space geometry}

A sphere, with a planar cut-off.
\begin{figure}[!h]
\begin{center}
\includegraphics[width=0.6\columnwidth]{Figures/sphere}
%\caption{Sketch of the formfactor sphere. Left: front view, right: top view.}
\end{center}
\label{sphere}
\end{figure} 
\par
\paragraph{Parameters:}
\begin{itemize}
\item radius $R$
\item heigth $H$
\end{itemize}
\paragraph{Restrictions:}
\begin{itemize}
\item $H\leq 2R$
\end{itemize}
\paragraph{Properties:}
\begin{itemize}
\item volume $V=\pi R^3 \left[\dfrac{2}{3} + \dfrac{H-R}{R} - \dfrac{1}{3}\left(\dfrac{H-R}{R}\right)^3\right]$
\item particle surface seen from above $S = \left\{\begin{array}{ll} \pi R^2, & H > R \\
         \pi\left(2RH-H^2\right), & H < R \end{array}\right. .$
\item gyration radius along $z$ axis $R_g = \left\{\begin{array}{ll}  R, & H > R \\
         \sqrt{2RH-H^2}, & H < R \end{array}\right. .$
\end{itemize}


\subsection{Computing the formfactor}
The formfactor can be computed analytically upto a 1-dimensional quadrature:
\begin{equation}
F(\mathbf q, R, H) = \exp\left[iq_z\left(H-R\right)\right]\int\limits_{R-H}^{R}{2\pi R_z^2\frac{J_1(q_{||}R_z)}{q_{||}R_z}\exp\left[iq_zz\right]dz} 
\label{eq:ffsphere}
\end{equation}
with abbreviations
\begin{eqnarray}
q_{||} &=& \sqrt{q_x^2+q_y^2} \\
R_z &=& \sqrt{R^2-z^2} 
\end{eqnarray}

\subsection{Exemplary formfactor}
$|F|^2/V^2$, computed for $R=10$~nm and $H=13$~nm:
\begin{figure}[h]
\begin{center}
\includegraphics[width=0.9\textwidth]{Figures/figffsphere}
\end{center}
%\caption{Computed results for $R=10$~nm and $H=13$~nm. {\bf a:} $|F|^2$ plotted against $q_z$ and $q_y$, {\bf b:} $|F|^2$ plotted against $q_x$ and $q_y$, {\bf c:} slice of the picture {\bf a} along the $q_z$ axis, {\bf d:} slice of the picture {\bf b} along the $q_x$ axis.}
\end{figure}

\par

\subsection{Parameter dependence}
$|F|^2/V^2$, computed for $R=10$~nm and $H=5$, 10 and 15~nm:
\begin{figure}[h]
\begin{center}
\includegraphics[width=0.9\textwidth]{Figures/ffspherepardep}
\end{center}
%\caption{Computed results for $R=10$~nm and different values of $H$.}
\end{figure}


\subsection{Related particle shapes}
Full sphere~(\SecRef{FullSphere}), Ellipsoid, Full spheroid, Hemi-ellipsoid.
%\begin{itemize}
%\item Full sphere, \SecRef{FullSphere}
%\item Ellipsoid
%\item Full spheroid
%\item Hemi-ellipsoid
%\end{itemize} 

\subsection{References}
Equation~(\ref{eq:ffsphere}) agrees with the \lq\lq sphere\rq\rq ~formfactor of \IsGISAXS~\cite{Laz02}.

\newpage{\cleardoublepage}
\section{Formfactor Pyramid}\SecLabel{Pyramid}

\subsection{Real-space geometry}

A truncated square pyramid.
\begin{figure}[!h]
\begin{center}
\includegraphics[width=0.6\columnwidth]{Figures/pyramid}
%\caption{Sketch of the formfactor sphere. Left: front view, right: top view.}
\end{center}
\label{pyramid}
\end{figure} 
\par
\paragraph{Parameters:}
\begin{itemize}
\item size of the basemend side $2R$
\item heigth $H$
\item angle $\alpha$
\end{itemize}
\paragraph{Restrictions:}
\begin{itemize}
\item $\dfrac{H}{R} < \tan(\alpha)$
\end{itemize}
\paragraph{Properties:}
\begin{itemize}
\item volume $V=\dfrac{4}{3}\tan(\alpha)\left[R^3-\left(R-\dfrac{H}{\tan(\alpha)}\right)^3\right]$
\item particle surface seen from above $S = 4R^2$
\item gyration radius along $z$ axis $R_g = \sqrt{2}R$
\end{itemize}


\subsection{Computing the formfactor}
The formfactor can be computed analytically upto a 1-dimensional quadrature:
\begin{equation}
F(\mathbf q, R, H,\alpha) = \int\limits_{0}^{H}{4R_z^2\sin_c(q_yR_z)\exp(iq_zz)dz} 
\label{eq:ffpyramid}
\end{equation}
or
\begin{equation}
\begin{aligned}
 & F(\mathbf q, R, H,\alpha) = \frac{H}{q_xq_y}\times \\  
 & \left[ \cos[(q_x-q_y)R]K_1 + \sin[(q_x-q_y)]K_2 - \cos[(q_x+q_y)R]K_3 - \sin[(q_x+q_y)R]K_4 \right] 
\end{aligned}
\end{equation}
with abbreviations
\begin{eqnarray}
R_z &=& R - \frac{z}{\tan(\alpha)} \\
K_1 &=& \sin_c(q_1H)\exp(iq_1H) + \sin_c(q_2H)\exp(-iq_2H) \\ 
K_2 &=& i\sin_c(q_1H)\exp(iq_1H) - i\sin_c(q_2H)\exp(-iq_2H) \\ 
K_3 &=& \sin_c(q_3H)\exp(iq_3H) + \sin_c(q_4H)\exp(-iq_4H) \\ 
K_4 &=& i\sin_c(q_3H)\exp(iq_3H) - i\sin_c(q_4H)\exp(-iq_4H) \\
q_1 &=& \frac{1}{2}\left[\frac{q_x-q_y}{\tan(\alpha)} + q_z\right] \\ 
q_2 &=& \frac{1}{2}\left[\frac{q_x-q_y}{\tan(\alpha)} - q_z\right] \\
q_3 &=& \frac{1}{2}\left[\frac{q_x+q_y}{\tan(\alpha)} + q_z\right] \\
q_4 &=& \frac{1}{2}\left[\frac{q_x+q_y}{\tan(\alpha)} - q_z\right] 
\end{eqnarray}

\subsection{Exemplary formfactor}
$|F|^2/V^2$, computed for $R=10$~nm, $H=13$~nm and $\alpha=60^{\circ}$:
\begin{figure}[h]
\begin{center}
\includegraphics[width=0.9\textwidth]{Figures/figffpyramid}
\end{center}
%\caption{Computed results for $R=10$~nm and $H=13$~nm. {\bf a:} $|F|^2$ plotted against $q_z$ and $q_y$, {\bf b:} $|F|^2$ plotted against $q_x$ and $q_y$, {\bf c:} slice of the picture {\bf a} along the $q_z$ axis, {\bf d:} slice of the picture {\bf b} along the $q_x$ axis.}
\end{figure}

\par

\subsection{Parameter dependence}
$|F|^2/V^2$, computed for $R=5$~nm, $\alpha=60^{\circ}$ and $H=2.5$, 5 and 7.5~nm:
\begin{figure}[h]
\begin{center}
\includegraphics[width=0.9\textwidth]{Figures/ffpyramidpardeph}
\end{center}
%\caption{Computed results for $R=10$~nm and different values of $H$.}
\end{figure}

$|F|^2/V^2$, computed for $R=5$~nm, $H=5$~nm and $\alpha=50^{\circ}$, $65^{\circ}$ and $80^{\circ}$:
\begin{figure}[h]
\begin{center}
\includegraphics[width=0.9\textwidth]{Figures/ffpyramidpardepa}
\end{center}
%\caption{Computed results for $R=10$~nm and different values of $H$.}
\end{figure}

\subsection{Related particle shapes}
Cubooctaedron, Anisotropic pyramid.

\subsection{References}
Equation~(\ref{eq:ffpyramid}) agrees with the \lq\lq pyramid\rq\rq ~formfactor of \IsGISAXS~\cite{Laz02}.


\newpage{\cleardoublepage}


\section{Formfactor Full Sphere}\SecLabel{FullSphere}
%\begin{lstlisting}[language=python, style=eclipseboxed,name=fffullsphere,nolol]
%from libBornAgainCore import * 
%fullsphere_ff = FormFactorFullSphere(5*nanometer)
%\end{lstlisting}
%\par
%~
%\par
Formfactor full sphere is represented as a sphere with radius $R$. 
\begin{figure}[ht]
\begin{center}
\includegraphics[width=0.6\columnwidth]{Figures/fullsphere}
\caption{Sketch of the formfactor full sphere. Left: front view, right: top view.}
\end{center}
\label{fullsphere}
\end{figure}

\begin{equation}
F(\mathbf q, R) = 4\pi R\times\frac{\sin(qR)-qR\cos(qR)}{(qR)^3}\times\exp\left(iq_zR\right)
\end{equation}

\begin{figure}[h]
\begin{center}
\includegraphics[width=0.9\textwidth]{Figures/figfffullsphere}
\end{center}
\caption{{\bf a:} $|F|^2$ plotted against $q_z$ and $q_y$, {\bf b:} $|F|^2$ plotted against $q_x$ and $q_y$, {\bf c:} slice of the picture {\bf a} along the $q_z$ axis, {\bf d:} slice of the picture {\bf b} along the $q_x$ axis.}
\end{figure}

\par

