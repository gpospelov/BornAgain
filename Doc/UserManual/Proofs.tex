%%%%%%%%%%%%%%%%%%%%%%%%%%%%%%%%%%%%%%%%%%%%%%%%%%%%%%%%%%%%%%%%%%%%%%%%%%%%%%%%
%%
%%   BornAgain User Manual
%%
%%   homepage:   http://www.bornagainproject.org
%%
%%   copyright:  Forschungszentrum Jülich GmbH 2015
%%
%%   license:    Creative Commons CC-BY-SA
%%   
%%   authors:    Scientific Computing Group at MLZ Garching
%%               C. Durniak, M. Ganeva, G. Pospelov, W. Van Herck, J. Wuttke
%%
%%%%%%%%%%%%%%%%%%%%%%%%%%%%%%%%%%%%%%%%%%%%%%%%%%%%%%%%%%%%%%%%%%%%%%%%%%%%%%%%

\chapter{Some proofs}  \label{sec:Proofs}

This appendix contains proofs that were taken out of the main text
in order not to disrupt the physics narration.

%%%%%%%%%%%%%%%%%%%%%%%%%%%%%%%%%%%%%%%%%%%%%%%%%%%%%%%%%%%%%%%%%%%%%%%%%%%%%%%%
\section{Source--detector reciprocity for scalar waves}
  \label{Sreci1}
%%%%%%%%%%%%%%%%%%%%%%%%%%%%%%%%%%%%%%%%%%%%%%%%%%%%%%%%%%%%%%%%%%%%%%%%%%%%%%%%

We derive a source-detector reciprocity theorem
\index{Reciprocity|(}%
\index{Green function!reciprocity}%
for the scalar Schrödinger equation.
It is needed in the derivation of the distorted-wave Born approximation
(Sect.~\ref{SDWBA}),
where it allows us to short-cut the computation of the Green function,
yielding at once the far-field at the detector position.

We start from a generic stationary Schrödinger equation
with an isolated inhomogeneity,
\begin{equation}\label{EgsSchrodi}
  \left\{\Nabla^2+v(\r)\right\}G(\r,\rS) = \delta(\r-\rS).
\end{equation}
We assume that the source location $\rS$
%\nomenclature[2r041 2s100]{$\rS$}{Position of a scattering center}%
(which in our application is a scattering center)
lies within a finite sample volume.
Outside the sample, the potential~$v(\r)$ has the constant value~$K^2$
so that \cref{EgsSchrodi}
reduces to the Helmholtz equation%
\begin{equation}\label{EgsSchrodiHelmh}
  \left\{\Nabla^2+K^2\right\}G(\r,\rS) = 0.
\end{equation}
We introduce the adjoint Green function~$B$
%\nomenclature[2b030 2r040 2r041]{$B(\r,\r')$}{Green function, adjoint of $G$}%
that originates from a source term at the detector location
and obeys
\begin{equation}\label{EgsSchrodiAdj}
  \left\{\Nabla^2+v(\r)\right\}B(\r,\rD) = \delta(\r-\rD).
\end{equation}
We also introduce the auxiliary vector field
\begin{equation}
  \v{X}(\r,\rS,\rD)\coloneqq B(\r,\rD)\Nabla G(\r,\rS) - G(\r,\rS)\Nabla B(\r,\rD).
\end{equation}
%\nomenclature[2x150 2r040]{$\v{X}(\r,\rS,\rD)$}{Auxiliary vector field}%
We inscribe the sample, the detector, and the origin of the coordinate system
into a sphere $\Sphere$ with radius~$R$,
%\nomenclature[2s180]{$\Sphere$}{Auxiliary spherical volume}%
%\nomenclature[2r120]{$R$}{Radius of $\Sphere$}%
and compute the volume integral
\begin{equation}\label{Eprerecipro}
  \begin{array}{lcl}
    I(\rS,\rD)
  &=& \displaystyle\int_\Sphere\!\d^3r\,\Nabla \v{X}(\r,\rS,\rD)
  \\[1.8em]
  &=& \displaystyle\int_\Sphere\!\d^3r\,\left(
    B\Nabla^2 G- G\Nabla^2 B \right)
  \\[1.4em]
  &=&  B(\rS,\rD) - G(\rD,\rS).
  \end{array}
\end{equation}
Alternatively, we can compute $I$ as a surface integral
\begin{equation}
  I(\rS,\rD)
  =\displaystyle\int_{\partial\Sphere}\d\v{\sigma}\,\v{X}(\r,\rS,\rD)
  =\displaystyle\int_{\partial\Sphere}\d{\sigma}\,
       \left(B\partial_R G - G\partial_R B\right).
\end{equation}
On the surface $\partial\Sphere$,
$B$ and $G$ are outgoing wave fields that obey the Helmholtz equation.
Solutions of this equation in spherical coordinates
have a well-known series expansion.
We send $R\to\infty$ so that we need only to retain the lowest order,
the form of which has been anticipated
in the boundary condition~\cref{Escabouco},
\begin{equation}
   G(\r(R,\vartheta,\varphi),\rS)
   \doteq \frac{\e^{iKR}}{4\pi R} g(\vartheta,\varphi),
\end{equation}
and similarly 
\begin{equation}
   B(\r(R,\vartheta,\varphi),\rD)
   \doteq \frac{\e^{iKR}}{4\pi R} b(\vartheta,\varphi).
\end{equation}
The functions $g$ and $b$ can be further expanded into spherical harmonics,
but this is of no interest here.
The decisive point is the factorization of $G$ and $B$
and their common $R$ dependence.
It follows at once that
\begin{equation}
  I(\rS,\rD)
  =\displaystyle\int_{\partial\Sphere}\d\sigma\,
       (\text{$R$-dependent})(bg-gb)
  = 0.
\end{equation}
From \cref{Eprerecipro} we obtain the \textit{reciprocity theorem}
\begin{equation}\label{Ereci}
  G(\rD,\rS) = B(\rS,\rD).
\end{equation}
It allows us to obtain the far-field value of the
forward-propagating Green function~$G$
at the detector position~$\rD$
from the adjoint Green function~$B$
that traces the radiation back from $\rD$
to the source location~$\rS$.
The theorem is practically important because
$B$ is much easier to compute than the unexpanded~$G$.

\index{Reciprocity|)}%
