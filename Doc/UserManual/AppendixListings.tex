%\newpage{\pagestyle{empty}\cleardoublepage}


\chapter{Listings}

%%%%%%%%%%%%%%%%%%%%%%%%%%%%%%%%%%%%%%%%%%%%%%%%%%%%%%%%%%%%%%%%%%%%%%%%%%%%%%%%%%%%%%%%%%%%%%%%%%
%
%%%%%%%%%%%%%%%%%%%%%%%%%%%%%%%%%%%%%%%%%%%%%%%%%%%%%%%%%%%%%%%%%%%%%%%%%%%%%%%%%%%%%%%%%%%%%%%%%%
\section{Python simulation example} \label{PythonSimulationExampleScript}
The following script can be found at
\begin{lstlisting}[language=shell, style=commandline]
./Examples/python/simulation/ex01_BasicParticles/CylindersAndPrisms.py
\end{lstlisting}


\begin{lstlisting}[
language=python, 
style=eclipse, 
escapeinside={@}{@},
frame = leftline, 
rulecolor = \color{lightgrey},
breaklines = true
]
import numpy
import matplotlib
import pylab
from libBornAgainCore import *


def get_sample():
    """
    Build and return the sample representing cylinders and pyramids on top of
    substrate without interference.
    """
    # defining materials
    m_air = HomogeneousMaterial("Air", 0.0, 0.0)
    m_substrate = HomogeneousMaterial("Substrate", 6e-6, 2e-8)
    m_particle = HomogeneousMaterial("Particle", 6e-4, 2e-8)

    # collection of particles
    cylinder_ff = FormFactorCylinder(5*nanometer, 5*nanometer)
    cylinder = Particle(m_particle, cylinder_ff)
    prism_ff = FormFactorPrism3(10*nanometer, 5*nanometer)
    prism = Particle(m_particle, prism_ff)
    particle_layout = ParticleLayout()
    particle_layout.addParticle(cylinder, 0.0, 0.5)
    particle_layout.addParticle(prism, 0.0, 0.5)
    interference = InterferenceFunctionNone()
    particle_layout.addInterferenceFunction(interference)

    # air layer with particles and substrate form multi layer
    air_layer = Layer(m_air)
    air_layer.addLayout(particle_layout)
    substrate_layer = Layer(m_substrate, 0)
    multi_layer = MultiLayer()
    multi_layer.addLayer(air_layer)
    multi_layer.addLayer(substrate_layer)
    return multi_layer


def get_simulation():
    """
    Create and return GISAXS simulation with beam and detector defined
    """
    simulation = Simulation()
    simulation.setDetectorParameters(100, -1.0*degree, 1.0*degree, 100, 0.0*degree, 2.0*degree)
    simulation.setBeamParameters(1.0*angstrom, 0.2*degree, 0.0*degree)
    return simulation


def run_simulation():
    """
    Run simulation and plot results
    """
    sample = get_sample()
    simulation = get_simulation()
    simulation.setSample(sample)
    simulation.runSimulation()
    result = simulation.getIntensityData().getArray() + 1  # for log scale
    pylab.imshow(numpy.rot90(result, 1), norm=matplotlib.colors.LogNorm(), extent=[-1.0, 1.0, 0, 2.0])
    pylab.show()


if __name__ == '__main__':
    run_simulation()

\end{lstlisting}


%%%%%%%%%%%%%%%%%%%%%%%%%%%%%%%%%%%%%%%%%%%%%%%%%%%%%%%%%%%%%%%%%%%%%%%%%%%%%%%%%%%%%%%%%%%%%%%%%%
%
%%%%%%%%%%%%%%%%%%%%%%%%%%%%%%%%%%%%%%%%%%%%%%%%%%%%%%%%%%%%%%%%%%%%%%%%%%%%%%%%%%%%%%%%%%%%%%%%%%
\newpage
\section{Python fitting example} \label{PythonFittingExampleScript}

The following script can be found at
\begin{lstlisting}[language=shell, style=commandline]
./Examples/python/fitting/ex02_FitCylindersAndPrisms/FitCylindersAndPrisms.py
\end{lstlisting}

\begin{lstlisting}[
language=python, 
style=eclipse, 
frame = leftline, 
rulecolor = \color{lightgrey},
breaklines = true
]
from libBornAgainCore import *
from libBornAgainFit import *


def get_sample():
    """
    Build the sample representing cylinders and pyramids on top of substrate without interference.
    """
    # defining materials
    m_air = HomogeneousMaterial("Air", 0.0, 0.0)
    m_substrate = HomogeneousMaterial("Substrate", 6e-6, 2e-8)
    m_particle = HomogeneousMaterial("Particle", 6e-4, 2e-8)

    # collection of particles
    cylinder_ff = FormFactorCylinder(1.0*nanometer, 1.0*nanometer)
    cylinder = Particle(m_particle, cylinder_ff)
    prism_ff = FormFactorPrism3(2.0*nanometer, 1.0*nanometer)
    prism = Particle(m_particle, prism_ff)
    particle_layout = ParticleLayout()
    particle_layout.addParticle(cylinder, 0.0, 0.5)
    particle_layout.addParticle(prism, 0.0, 0.5)
    interference = InterferenceFunctionNone()
    particle_layout.addInterferenceFunction(interference)

    # air layer with particles and substrate form multi layer
    air_layer = Layer(m_air)
    air_layer.addLayout(particle_layout)
    substrate_layer = Layer(m_substrate, 0)
    multi_layer = MultiLayer()
    multi_layer.addLayer(air_layer)
    multi_layer.addLayer(substrate_layer)
    return multi_layer


def get_simulation():
    """
    Create GISAXS simulation with beam and detector defined
    """
    simulation = Simulation()
    simulation.setDetectorParameters(100, -1.0*degree, 1.0*degree, 100, 0.0*degree, 2.0*degree)
    simulation.setBeamParameters(1.0*angstrom, 0.2*degree, 0.0*degree)
    return simulation


def run_fitting():
    """
    run fitting
    """
    sample = get_sample()
    simulation = get_simulation()
    simulation.setSample(sample)

    real_data = IntensityDataIOFactory.readIntensityData('refdata_fitcylinderprisms.int.gz')
    
    fit_suite = FitSuite()
    fit_suite.addSimulationAndRealData(simulation, real_data)
    fit_suite.initPrint(10)

    # setting fitting parameters with starting values
    fit_suite.addFitParameter("*FormFactorCylinder/height", 4.*nanometer, 0.01*nanometer, AttLimits.lowerLimited(0.01))
    fit_suite.addFitParameter("*FormFactorCylinder/radius", 6.*nanometer, 0.01*nanometer, AttLimits.lowerLimited(0.01))
    fit_suite.addFitParameter("*FormFactorPrism3/height", 4.*nanometer, 0.01*nanometer, AttLimits.lowerLimited(0.01))
    fit_suite.addFitParameter("*FormFactorPrism3/length", 12.*nanometer, 0.02*nanometer, AttLimits.lowerLimited(0.01))

    # running fit
    fit_suite.runFit()
    
    print "Fitting completed."
    fit_suite.printResults()
    print "chi2:", fit_suite.getMinimizer().getMinValue()
    fitpars = fit_suite.getFitParameters()
    for i in range(0, fitpars.size()):
        print fitpars[i].getName(), fitpars[i].getValue(), fitpars[i].getError()

if __name__ == '__main__':
    run_fitting()
\end{lstlisting}

